\documentclass{article}

%other packages
\usepackage[a4paper]{geometry}
\usepackage{longtable}
\usepackage{wrapfig}
\setlength\parindent{0pt}
\usepackage{enumitem}
\usepackage[table]{xcolor}
\usepackage{polynom}
\def\scaleint#1{\vcenter{\hbox{\scaleto[3ex]{\displaystyle\int}{#1}}}}
\usepackage{array}
\newcolumntype{C}{>{{}}c<{{}}} % for '+' and '-' symbols
\newcolumntype{R}{>{\displaystyle}r} % automatic display-style math mode 
\usepackage{tabularray}
\usepackage{dcolumn,tabularx,booktabs}
\usepackage{esvect}

%maths
\usepackage{mathtools}
\usepackage{amsmath}
\usepackage{amssymb}
\usepackage{amsfonts}
\usepackage{autobreak}

%tikzpicture
\usepackage{tikz}
\usepackage{scalerel}
\usepackage{pict2e}
\usepackage{tkz-euclide}
\usepackage{tikz-3dplot}
\usetikzlibrary{calc}
\usetikzlibrary{patterns,arrows.meta}
\usetikzlibrary{shadows}
\usetikzlibrary{external}
\usetikzlibrary{decorations.pathreplacing,angles,quotes}
\usetikzlibrary{perspective,spath3}

%pgfplots
\usepackage{pgfplots}
\pgfplotsset{compat=1.18}
\usepgfplotslibrary{statistics}
\usepgfplotslibrary{fillbetween}

\pgfplotsset{
    standard/.style={
    axis line style = thick,
    trig format=rad,
    enlargelimits,
    axis x line=middle,
    axis y line=middle,
    enlarge x limits=0.15,
    enlarge y limits=0.15,
    every axis x label/.style={at={(current axis.right of origin)},anchor=north west},
    every axis y label/.style={at={(current axis.above origin)},anchor=south east}
    }
}

\begin{document}

Math 115 - Week 6, Class 14 - 5 Feb 2024
\hrule

\vspace{10pt}

We are integrating functions of the form:

$R(x)=\mbox{ polynomial }+\mbox{ proper fraction }=\mbox{ polynomial }+\sum\mbox{ (partial fractions) }$

\vspace{10pt}

{\bf{}EXAMPLE} Evaluate $\displaystyle\int\frac{3x^2+1}{x^3+1}\ dx$

\vspace{10pt}

First we use partial fraction decomposition to re-express the integrand as the sum of its partial fractions. Recall the formula $(a^3+b^3)=(a+b)(a^2-ab+b^2)$

\begin{align*}
\frac{3x^2+1}{(x+1)(x^2-x+1)}&=\frac{A}{x+1}+\frac{Bx+C}{x^2-x+1}\\
&=\frac{A(x^2-x+1)+(Bx+C)(x+1)}{(x+1)(x^2-x+1)}\\
&=\frac{(A+B)x^2+(-A+B+C)x+(A+C)}{(x+1)(x^2-x+1)}
\end{align*}

Now we can obtain a system of linear equations. We could use more advanced techniques, such as Cramer's Rule or Gaussian Elimination, but let's do it the way Our Math Professor showed us.

\[\left\{\begin{aligned}A+B=3\\-A+B+C=0\\A+C=1\end{aligned}\right.\]

\begin{align*}
&2A+B+C-(-A+B+C)=4\\
&3A=4\quad\boxed{A=\frac{4}{3}}\\
&B=3-\frac{4}{3}=\frac{8}{3}\\
&C=1-\frac{4}{3}=\frac{-1}{3}
\end{align*}

This lets us rewrite the integral;

\begin{align*}
&\Rightarrow\frac{4}{3}\int\frac{1}{x+1}\ dx+\frac{1}{3}\int\frac{5x-1}{x^2-x+1}\ dx\\
&=\left(\begin{aligned}\mbox{ completed square }=\left(x-\frac{1}{2}\right)^2+\frac{3}{4}\\x-\frac{1}{2}=t\quad dx=\ dt\quad x=t+\frac{1}{2}\end{aligned}\right)\\
&=\frac{4}{3}\ln|x+1|+\frac{1}{3}\int\frac{5t+\frac{5}{2}-1}{t^2+\frac{3}{4}}\ dt\\
&=\frac{4}{3}\ln|x+1|+\frac{1}{3}\left[\int\frac{t}{t^2+\frac{3}{4}}\ dt+\int\frac{3/2}{t^2+\frac{3}{4}}\ dt\right]
\end{align*}

The following formula is easy to derive using the difference of squares formula and partial fractions; that being said, Our Math Professor thinks it is useful to remember. I choose to do it by derivation because I can't remember all these formulas, but it's up to you.

\begin{center}
$\displaystyle\boxed{\int\frac{1}{x^2-a^2}\ dx=\frac{1}{2}\ln\left|\frac{x-a}{x+a}\right|+C}$
\end{center}

\vspace{10pt}

We then went over the fourth (and final) partial fraction decomposition technique - for denominators which contain powers of irreducible quadratics.

\begin{align*}
\int\frac{Ax+B}{(x^2-px+q)^k}\ dx&\overset{\mbox{complete}}{\underset{\mbox{square}}{=\joinrel=\joinrel=\joinrel=\joinrel=\joinrel=\joinrel=}}\int\frac{At+C}{(t^2+a^2)^k}\ dt\\
&=A\int\frac{t}{(t^2+a^2)^k}\ dt+C\int\frac{1}{(t^2+a^2)^k}\ dt\\
\end{align*}

\vspace{10pt}

Both terms are worthy of their own questions, tbh. So, we will treat them separately.

\begin{align*}
A\int\frac{t}{(t^2+a^2)^k}\ dt&=\left(\begin{aligned}t\ dt=\frac{1}{2}\ d(t^2+a^2)\\t^2+a^2=u\end{aligned}\right)\\
&=\frac{1}{2}\int\frac{du}{u^k}=\frac{1}{2}\frac{u^{-k+1}}{-k+1}+C\\
&\overset{\mbox{back}}{\underset{\mbox{substitute}}{=\joinrel=\joinrel=\joinrel=\joinrel=\joinrel=\joinrel=}}\frac{1}{2}\frac{(t^2+a^2)^{-k+1}}{-k+1}+C
\end{align*}

\vspace{10pt}

Then we did the second one;

\begin{align*}
\int\frac{1}{(t^2+a^2)^k}\ dt&=\frac{1}{a^2}\int\frac{a^2}{(t^2+a^2)^k}\ dt\\
&=\frac{1}{a^2}\int\frac{a^2+t^2-t^2}{(t^2+a^2)^k}\ dt\\
&=\frac{1}{a^2}\left[\int(t^2+a^2)^{-k+1}\ dt-\int\frac{t^2}{t^2+a^2}\ dt\right]\\
&=\left(\begin{aligned}&t=u\quad du=\ dt\\&dV=\frac{t\ dt}{(t^2+a^2)^k}\quad V\begin{aligned}[t]&=\frac{1}{2}\int\frac{d\ (t^2+a^2)}{(t^2+a^2)^k}\\&=\frac{1}{2}(t^2+a^2)^{-k+1}\end{aligned}\end{aligned}\right)\\
&=\left[t\cdot\frac{1}{2}\frac{(t^2+a^2)^{-k+1}}{-k+1}-\frac{1}{2(-k+1)}\cdot\int\frac{dt}{(t^2+a^2)^{k-1}}\right]\\
&=\left[\frac{t}{2(1-k)}\cdot(t^2+a^2)^{1-k}-\frac{1}{2(-k+1)}\cdot\int(t^2+a^2)^{-k+1}\ dt\right]\\
&\mbox{So, }\begin{aligned}[t]&\int\frac{1}{(t^2+a^2)^k}\ dt=\frac{1}{a^2}\left[\int(t^2+a^2)^{-k+1}\ dt\right.\\&\left.-\left[\frac{t}{2(1-k)}\cdot(t^2+a^2)^{1-k}-\frac{1}{2(-k+1)}\cdot\int(t^2+a^2)^{-k+1}\ dt\right]\right]\end{aligned}
\end{align*}

\vspace{10pt}

And Our Math Professor mentioned integration by trigonometric substitution

(Friendly heads up: if you haven't \textit{mastered} the trig formulas yet, \textit{now} is the time.)


\end{document}